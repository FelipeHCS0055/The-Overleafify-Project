\chapter{Bibliografia e citações em LaTeX}

\chapterquote{``Citação.''}{Autor}

% --------

\section{Introdução ao gerenciamento de referências}

O LaTeX oferece ferramentas poderosas para criar e gerenciar referências bibliográficas de forma automatizada, garantindo consistência e padronização em documentos acadêmicos.

\section{Citações básicas}

Use o comando \verb|\cite{citação}| para citar uma referência.

\begin{lstlisting}[language=tex, caption=Exemplo simples de citação]
    Segundo \cite{einstein}, a teoria da relatividade...
\end{lstlisting}

\section{Criando uma bibliografia manual}

Para documentos pequenos, use o ambiente \verb|thebibliography|:

\begin{lstlisting}[language=tex, caption=Equação inline]
    \begin{thebibliography}{9} % O número 9 indica a largura máxima de rótulos
        \bibitem{einstein}  
            Einstein, A. (1905).  
            \textit{Zur Elektrodynamik bewegter Körper}.  
            Annalen der Physik, 17(10), 891–921.  
    \end{thebibliography}
\end{lstlisting}

\section{Usando o BibTeX (Recomendado)}

O \textbf{BibTeX} automatiza a criação de bibliografias. Siga os passos:

\begin{enumerate}
    \item Crie um arquivo \verb|.bib| (ex: \verb|referencias.bib|) com entradas no formato:
    \begin{lstlisting}[language=tex, caption=Exemplo de formatação em um arquivo \texttt{.bib}]
    @article{einstein1905,
        author  = "Albert Einstein",
        title   = "Zur Elektrodynamik bewegter Körper",
        journal = "Annalen der Physik",
        year    = "1905",
        volume  = "322",
        pages   = "891--921"
    }
    \end{lstlisting}

    \item Inclua o arquivo no documento:
    \begin{lstlisting}[language=tex, caption=Incluindo o arquivo de bibliografias no documento]
    \bibliographystyle{plain} % Estilo (ex: abnt, ieee, apa)
    \bibliography{referencias} % Nome do arquivo .bib (sem extensão)
    \end{lstlisting}
\end{enumerate}

\section{Estilos de citação}

Altere o estilo com \verb|\bibliographystyle|:

\begin{itemize}
    \item \verb|plain|: Numérico (padrão).
    \item \verb|abnt|: Normas ABNT (requer o pacote \verb|abntex2|).
    \item \verb|apa|: Formato APA.
    \item \verb|ieee|: Padrão IEEE.
\end{itemize}

\begin{lstlisting}[language=tex, caption=Definindo o estilo da bibliografia]
    \bibliographystyle{ieee}  
    \bibliography{referencias}  
\end{lstlisting}

\section{Dicas práticas}
\begin{itemize}
    \item \textbf{Zotero + Better BibTeX}: Exporte referências diretamente para \verb|.bib|.
    \item \textbf{JabRef}: Editor gratuito para gerenciar arquivos \verb|.bib|.
    \item \textbf{DOI para BibTeX}: Sites como \href{https://www.doi2bib.org/}{doi2bib} convertem DOIs em entradas BibTeX.
    \item Erros comuns:
    \begin{itemize}
        \item Referência não aparece: Verifique se a chave em \verb|\cite{citação}| corresponde à entrada no arquivo \verb|.bib|.
        \item Estilo incorreto: Confira se o estilo desejado e adequado está instalado.
    \end{itemize}
\end{itemize}
