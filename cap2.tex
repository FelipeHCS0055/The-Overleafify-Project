\chapter{Texto e formatação}

\chapterquote{``Citação.''}{Autor}

% --------

\section{Formatação básica}

\begin{itemize}
    \item \textbf{Negrito}: \verb|\textbf{Texto em negrito}|
    \item \textit{Itálico}: \verb|\textit{Texto em itálico}|
    \item \underline{Sublinhado}: \verb|\underline{Texto sublinhado}|
\end{itemize}

\begin{lstlisting}[language=tex, caption=Formatação básica de texto]
    \textbf{Importante}: \textit{Não esqueça} de \underline{revisar} o código.
\end{lstlisting}

O código acima gera como saída:\\

\textbf{Importante}: \textit{Não esqueça} de \underline{revisar} o código.

\section{Tamanhos da fonte}

Tamanhos pré-definidos (do menor ao maior):
{\tiny Texto pequeno}  
{\scriptsize Um pouco maior}  
{\footnotesize Notas de rodapé}  
{\small Pequeno}  
{\normalsize Normal}  
{\large Grande}  
{\Large Maior}  
{\LARGE Muito grande}  
{\huge Enorme}  
{\Huge Gigante}

\begin{lstlisting}[language=tex, caption=Tamanhos de fontes pré-definidos]
    {\tiny Texto pequeno}  
    {\scriptsize Um pouco maior}  
    {\footnotesize Notas de rodapé}  
    {\small Pequeno}  
    {\normalsize Normal}  
    {\large Grande}  
    {\Large Maior}  
    {\LARGE Muito grande}  
    {\huge Enorme}  
    {\Huge Gigante}  
\end{lstlisting}

Dica: Use \verb|\normalsize| para voltar ao tamanho padrão após alterações.

\section{Cores no texto}

Use o pacote \verb|xcolor|:

\begin{lstlisting}[language=tex, caption=Uso do pacote \texttt{xcolor} para customização de cores no texto]
    \usepackage{xcolor}  
    \textcolor{red}{Texto vermelho}  
    \textcolor{blue}{Texto azul}  
    \textcolor[HTML]{00FF00}{Verde em hexadecimal}  
\end{lstlisting}

\textcolor{red}{Texto vermelho}  
\textcolor{blue}{Texto azul}  
\textcolor[HTML]{00FF00}{Verde em hexadecimal}  

\section{Listas}

\subsection{Listas enumeradas - ambiente \texttt{enumerate}}

\begin{lstlisting}[language=tex, caption=Listas enumeradas]
    \begin{enumerate}
        \item Primeiro item
        \item Segundo item
    \end{enumerate} 
\end{lstlisting}

\begin{enumerate}
    \item Primeiro item
    \item Segundo item
\end{enumerate}

\subsection{Listas não enumeradas - ambiente \texttt{itemize}}

\begin{lstlisting}[language=tex, caption=Listas não enumeradas]
    \begin{itemize}
        \item Item com marcador padrão
        \item Outro item
    \end{itemize}
\end{lstlisting}

\begin{itemize}
    \item Item com marcador padrão
    \item Outro item
\end{itemize}

\subsection{Listas descritivas - ambiente \texttt{description}}

\begin{lstlisting}[language=tex, caption=Listas descritivas]
    \begin{description}
        \item[LaTeX] Sistema de preparação de documentos.
        \item[Overleaf] Plataforma online para LaTeX.
    \end{description}
\end{lstlisting}

\begin{description}
    \item[LaTeX] Sistema de preparação de documentos.
    \item[Overleaf] Plataforma online para LaTeX.
\end{description}

Dica: Use o pacote \verb|enumitem| para personalizar marcadores e espaçamento.

\section{Espaçamento e parágrafos}
\begin{itemize}
    \item Quebra de linha: \verb|\\| ou \verb|\newline|
    \item Novo parágrafo: Deixe uma linha vazia no código ou use \verb|\par|.
    \item Espaçamento vertical: \verb|\vspace{1cm}|
    \item Espaçamento horizontal: \verb|\hspace{2em}|
\end{itemize}

\begin{lstlisting}[language=tex, caption=Exemplo de espaçamento e parágrafos]
    Primeira linha.\\ Segunda linha.  
    \par Terceira linha (novo parágrafo).
\end{lstlisting}

Primeira linha.\\ Segunda linha.  
\par Terceira linha (novo parágrafo).

% \section{Espaçamento entre linhas}

% Use o pacote \verb|setspace|:

\section{Dicas práticas}

\begin{itemize}
    \item Pacotes úteis:
    \begin{lstlisting}[language=tex, caption=Outros pacotes úteis para texto e formatação]
    \usepackage{lipsum}   % Gerar texto aleatório para testes
    \usepackage{hyperref} % Links clicáveis
    \end{lstlisting}
    \item Erros comuns:
    \begin{itemize}
        \item Esquecer de fechar chaves \verb|{}| após comandos de formatação.
        \item Usar \verb|\\| excessivamente (pode causar layout quebrado).
    \end{itemize}
\end{itemize}

