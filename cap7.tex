\chapter{Layouts profissionais}

\chapterquote{``Citação.''}{Autor}

% --------

\section{Controle de geometria da página}

Use o pacote \verb|geometry| para ajustar margens, cabeçalhos e rodapés:

\begin{lstlisting}[language=tex, caption=Configuração da geometria das páginas]
    \usepackage[top=3cm, bottom=2.5cm, left=3cm, right=2cm, headheight=15pt]{geometry}
\end{lstlisting}    

Parâmetros úteis:
\begin{itemize}
    \item \verb|includeheadfoot|: Inclui cabeçalho/rodapé no cálculo das margens.
    \item \verb|landscape|: Modo paisagem (\verb|\usepackage{pdflscape}| para páginas específicas).
\end{itemize}

\section{Cabeçalhos e rodapés personalizados}

Use o pacote \verb|fancyhdr| para designs avançados:

\begin{lstlisting}[language=tex, caption=Usando o pacote \texttt{fancyhdr} para personalização de estilo]
    \usepackage{fancyhdr}
    \pagestyle{fancy}
    \fancyhf{} % Limpa estilos padrão
    \rhead{\textit{Nome do Documento}} 
    \lfoot{\thepage}
    \rfoot{\today}
    \renewcommand{\headrulewidth}{1pt} % Linha no cabeçalho
\end{lstlisting}   

Estilos diferentes para páginas específicas:

\begin{lstlisting}[language=tex, caption=Estilizando páginas específicas]
    \fancypagestyle{plain}{ % Estilo para páginas de capítulo
        \fancyhf{}
        \rfoot{Página \thepage}
    }
\end{lstlisting}   

\section{Personalização de títulos e seções}

\begin{lstlisting}[language=tex, caption=Pacote \texttt{titlesec} para alterar títulos e seções]
    \usepackage{titlesec}
    \titleformat{\section}[block] % Formato do título
        {\bfseries\Large\color{blue}} % Fonte
        {\thesection} % Rótulo
        {1em} % Espaço entre rótulo e título
        {} 
    
    \titlespacing{\section}{0pt}{12pt}{6pt} % Espaçamento: esquerda, acima, abaixo
\end{lstlisting}  

\section{Fontes e tipografia}

\begin{itemize}
    \item Pacotes de fontes:
    \begin{lstlisting}[language=tex, caption=Pacotes de fontes mais usados]
    \usepackage{lmodern}       % Latin Modern (padrão aprimorado)
    \usepackage{kpfonts}       % Fonte Kepler (estilo clássico)
    \usepackage{roboto}        % Fonte moderna (requer XeLaTeX/LuaLaTeX)
    \end{lstlisting} 
    
    \item Configuração global:
    \begin{lstlisting}[language=tex, caption=Fonte padrão para o documento]
    \renewcommand{\familydefault}{\sfdefault} % Muda fonte padrão para sans-serif
    \end{lstlisting}      
\end{itemize}

\section{Cores temáticas}

Use o pacote \verb|xcolor| para definir esquemas de cores:

\begin{lstlisting}[language=tex, caption=Alterando a cores de títulos e seções]
    \definecolor{corprimaria}{RGB}{0, 102, 204} % Azul institucional
    \definecolor{cortitulo}{HTML}{800080}       % Roxo em hexadecimal
    
    % Aplicação em títulos:
    \titleformat{\chapter}{\color{cortitulo}\Huge\bfseries}{\thechapter}{1em}{}
\end{lstlisting}

\section{Caixas e destaques}

Crie caixas estilizadas com \verb|mdframed| ou \verb|tcolorbox|:

\begin{lstlisting}[language=tex, caption=Exemplo de caixa de destaque]
    \usepackage{tcolorbox}
    \newtcolorbox{caixadestaque}{
        colback=blue!10!white, % Cor de fundo
        colframe=blue!50!black, % Cor da borda
        arc=4pt, % Arredondamento
        boxrule=1.5pt % Espessura da borda
    }
    
    % Uso:
    \begin{caixadestaque}
        Texto destacado aqui.
    \end{caixadestaque}
\end{lstlisting}

\newtcolorbox{caixadestaque}{
    colback=blue!10!white, % Cor de fundo
    colframe=blue!50!black, % Cor da borda
    arc=4pt, % Arredondamento
    boxrule=1.5pt % Espessura da borda
}

% Uso:
\begin{caixadestaque}
    Texto destacado aqui.
\end{caixadestaque}

\section{Ambientes personalizados}

Crie ambientes para teoremas, exemplos ou definições:

\begin{lstlisting}[language=tex, caption=Caixa de destaque simples para teorema]
    \usepackage{amsthm}
    \newtheorem{teorema}{Teorema}[section] % Numeração por seção
    
    % Uso:
    \begin{teorema}[Pitágoras]
        Em um triângulo retângulo, \(a^2 + b^2 = c^2\).
    \end{teorema}
\end{lstlisting}

\newtheorem{teorema1}{Teorema}[section] % Numeração por seção

% Uso:
\begin{teorema1}[Pitágoras]
    Em um triângulo retângulo, \(a^2 + b^2 = c^2\).
\end{teorema1}

Você pode se aprofundar bastante aqui. Veja os exemplos abaixo:

\begin{tcolorbox}[enhanced,
colback=red!10!white,colframe=red!75!black,
colbacktitle=red!50!yellow,fonttitle=\bfseries,
frame hidden,
title=My title,
boxrule=0pt,titlerule=1mm,
titlerule style=red!50!black ]
This is a \textbf{tcolorbox}.
\end{tcolorbox}

\begin{tcolorbox}[tile,flip title={sharp corners},
title=My title,colback=red!10,
colbacktitle=red!75!black]
This is a \textbf{tcolorbox}.
\end{tcolorbox}

\begin{tcolorbox}[enhanced,title=My title,
attach boxed title to bottom*]
This is a \textbf{tcolorbox}.
\end{tcolorbox} 

\begin{tcolorbox}[enhanced,title=My title,
attach boxed title to top center={yshift*=-3mm},
boxed title style={size=small,colback=blue}]
This is a \textbf{tcolorbox}.
\end{tcolorbox}

\begin{tcolorbox}[enhanced,title=My title,
colframe=red!50!black,colback=red!10!white,
arc=1mm,colbacktitle=red!10!white,
fonttitle=\bfseries,coltitle=red!50!black,
attach boxed title to top text left=
{yshift=-0.50mm},
boxed title style={skin=enhancedfirst jigsaw,
size=small,arc=1mm,bottom=-1mm,
interior style={fill=none,
top color=red!30!white,
bottom color=red!20!white}}]
This is a \textbf{tcolorbox}.
\end{tcolorbox}

% \usepackage{varwidth}
\newtcolorbox{mybox1}[2][]{enhanced,skin=enhancedlast jigsaw,
attach boxed title to top left={xshift=-4mm,yshift=-0.5mm},
fonttitle=\bfseries\sffamily,varwidth boxed title=0.7\linewidth,
colbacktitle=blue!45!white,colframe=red!50!black,
interior style={top color=blue!10!white,bottom color=red!10!white},
boxed title style={empty,arc=0pt,outer arc=0pt,boxrule=0pt},
underlay boxed title={
\fill[blue!45!white] (title.north west) -- (title.north east)
-- +(\tcboxedtitleheight-1mm,-\tcboxedtitleheight+1mm)
-- ([xshift=4mm,yshift=0.5mm]frame.north east) -- +(0mm,-1mm)
-- (title.south west) -- cycle;
\fill[blue!45!white!50!black] ([yshift=-0.5mm]frame.north west)
-- +(-0.4,0) -- +(0,-0.3) -- cycle;
\fill[blue!45!white!50!black] ([yshift=-0.5mm]frame.north east)
-- +(0,-0.3) -- +(0.4,0) -- cycle; },
title={#2},#1}
\begin{mybox1}{My title}
\lipsum[2]
\end{mybox1}

\newtcolorbox{mybox2}[1]{hbox boxed title,
enhanced,attach boxed title to top center=
{yshift=-3mm,yshifttext=-1mm},
boxed title style={size=small,colback=red},
title={#1}}
\begin{mybox2}{Short title}
This is a \textbf{tcolorbox}.
\end{mybox2}\bigskip
\begin{mybox2}{This title is not really very short}
This is a \textbf{tcolorbox}.
\end{mybox2}

\begin{tcolorbox}[drop shadow southeast,
enhanced,colback=red!5!white,colframe=red!75!black]
This is a tcolorbox.
\end{tcolorbox}

\newtcolorbox{mybox3}[1][]{enhanced,
fuzzy shadow={1.0mm}{-1.0mm}{0.12mm}{0mm}{blue!50!white},
fuzzy shadow={-1.0mm}{-1.0mm}{0.12mm}{0mm}{red!50!white},
fuzzy shadow={-1.0mm}{1.0mm}{0.12mm}{0mm}{green!50!white},
fuzzy shadow={1.0mm}{1.0mm}{0.12mm}{0mm}{yellow!50!white},#1
}
\begin{mybox3}[title=A multi shadow box]
This is a tcolorbox.
\end{mybox3}

Esses são apenas alguns exemplos retirados da \href{https://www.tug.org/docs/latex/tcolorbox/tcolorbox.pdf}{documentação do pacote \texttt{tcolorbox}}. Existem muitos outros modelos feitos pela comunidade, além de várias outras aplicações e usos desse pacote.

\section{Dicas práticas}

\begin{itemize}
    \item Consistência: Defina um arquivo \verb|.sty| com todas as personalizações para reutilização.
    \item Templates: Use \href{https://www.overleaf.com/latex/templates}{modelos do Overleaf} como base.
    \item Evite excessos: Priorize clareza e normas da instituição.
\end{itemize}