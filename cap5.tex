\chapter{Manipulação de Imagens em LaTeX}

\chapterquote{``Citação.''}{Autor}

% --------

\section{Modos matemáticos}

\begin{itemize}
    \item Inline (dentro do texto): Use \verb|\(...\)| ou \verb|$...$|:
    \begin{lstlisting}[language=tex, caption=Equação inline]
    A equação \(E = mc^2\) foi proposta por Einstein.
    \end{lstlisting}
    Saída: A equação \(E = mc^2\) foi proposta por Einstein.
    \item Display: Use \verb|\[...\]| ou \verb|$$...$$|:
    \begin{lstlisting}[language=tex, caption=Equação display]
    A fórmula da energia é:
    \[E = mc^2\]
    \end{lstlisting}
    Saída: \[E = mc^2\]
\end{itemize}

\section{Símbolos básicos}

\begin{itemize}
    \item Letras gregas: \verb|\alpha|, \verb|\beta|, \verb|\gamma|, \verb|\Gamma|, \verb|\Delta|, \verb|\Omega|
    \(\alpha\), \(\beta\), \(\gamma\), \(\Gamma\), \(\Delta\), \(\Omega\)
    \item Operadores e relações: \verb|\times| (multiplicação), \verb|\pm| (mais/menos), \verb|\leq| (menor ou igual), \verb|\geq| (maior ou igual)
    \(\times\), \(\pm\), \(\leq\), \(\geq\)
    \item Setas e símbolos lógicos: \verb|\leftarrow|, \verb|\Leftarrow|, \verb|\rightarrow|, \verb|\Rightarrow|, \verb|\exists| (existe), \verb|\forall| (para todo)
    \(\leftarrow\), \(\Leftarrow\), \(\rightarrow\), \(\Rightarrow\), \(\exists\), \(\forall\)
\end{itemize}

\section{Frações, expoentes e índices}

\begin{itemize}
    \item Frações:
    \begin{lstlisting}[language=tex, caption=Frações em LaTeX]
    \begin{center}
        \(\frac{a}{b}\)\\
    \end{center}
    \[\dfrac{a}{b}\] (uso em display mode)
    \end{lstlisting}
    \begin{center}
        \(\frac{a}{b}\)\\
    \end{center}
    \[\dfrac{a}{b}\]
\end{itemize}

Use \verb|dfrac{a}{b}| para o modo display - o tamanho fica maior.

\section{Matrizes e vetores}

\begin{itemize}
    \item Matrizes simples:
    \begin{lstlisting}[language=tex, caption=Matriz simples]
    \[
    \begin{matrix}
    1 & 2 \\
    3 & 4
    \end{matrix}
    \]
    \end{lstlisting}
    \[
    \begin{matrix}
    1 & 2 \\
    3 & 4
    \end{matrix}
    \]

    \item Matrizes com parênteses (usando \verb|pmatrix|):
    \begin{lstlisting}[language=tex, caption=Matriz com parênteses]
    \[
    \begin{pmatrix}
    a & b \\
    c & d
    \end{pmatrix}
    \]
    \end{lstlisting}
    \[
    \begin{pmatrix}
    a & b \\
    c & d
    \end{pmatrix}
    \]    

    \item Matrizes com colchetes (usando \verb|bmatrix|):
    \begin{lstlisting}[language=tex, caption=Matriz com colchetes]
    \[
    \begin{bmatrix}
    X & Y \\
    Z & W
    \end{bmatrix}
    \]
    \end{lstlisting}
    \[
    \begin{bmatrix}
    X & Y \\
    Z & W
    \end{bmatrix}
    \] 
\end{itemize}

\section{Ambientes avançados}

\begin{itemize}
    \item Equações numeradas:
    \begin{lstlisting}[language=tex, caption=Equação numerada]
    \begin{equation}
        \label{eq:energia}
        E = mc^{2}
    \end{equation}
    \end{lstlisting}
    \begin{equation}
        \label{eq:energia}
        E = mc^{2}
    \end{equation}

    \item Alinhamento de equações (ambiente \verb|align|):
    \begin{lstlisting}[language=tex, caption=Equações alinhadas]
    \begin{align}
        x & + y = 5 \\
        2x - y = 3 &
    \end{align}
    \end{lstlisting}
    \begin{align}
        x & + y = 5 \\
        2x - y = 3 &
    \end{align}
\end{itemize}

Para não enumerar as equações, use o ambiente \verb|align*|:

\begin{lstlisting}[language=tex, caption=Equações alinhadas e não enumeradas]
\begin{align*}
    x & + y = 5 \\
    2x - y = 3 &
\end{align*}
\end{lstlisting}
\begin{align*}
    x & + y = 5 \\
    2x - y = 3 &
\end{align*}

\section{Sistemas de equações}

Use o ambiente \verb|cases| (requer o pacote \verb|amsmath|):

\begin{lstlisting}[language=tex, caption=Sistema de equações]
    \[
    f(x) = 
    \begin{cases}
    x^{2} & \text{se } x \geq 0 \\
    -x & \text{se } x < 0
    \end{cases}
    \]
\end{lstlisting}

\[
f(x) = 
\begin{cases}
x^{2} & \text{se } x \geq 0 \\
-x & \text{se } x < 0
\end{cases}
\]

\section{Dicas práticas}

\begin{itemize}
    \item Pacote essencial:
    \begin{lstlisting}[language=tex, caption=Carregue o pacote \texttt{amsmath}]
    \usepackage{amsmath} % Carregue sempre!
    \end{lstlisting}
    \item Erros comuns:
    \begin{itemize}
        \item Esquecer de usar \verb|\| antes de símbolos especiais (ex: \verb|\alpha|, não \verb|alpha|).
        \item Não fechar chaves em expoentes/subscritos (ex: \verb|x^{2}| em vez de \verb|x^2|).
    \end{itemize}
\end{itemize}

Dica: Use o \href{https://detexify.kirelabs.org/classify.html}{Detexify} para encontrar símbolos matemáticos desconhecidos!