\chapter{Tipos de documentos e estruturas}

\chapterquote{``Citação.''}{Autor}

% --------

\section{Classes de documentos no LaTeX}

O LaTeX oferece diferentes classes de documentos para atender a diversos propósitos. As principais são:
\begin{enumerate}
    \item \verb|article|: Ideal para artigos científicos, relatórios curtos e documentos simples.
    \begin{lstlisting}[language=tex]
    \documentclass[a4paper, 12pt]{article} % Tamanho A4, fonte 12pt
    \end{lstlisting}
    \item \verb|report|: Usado para relatórios técnicos, monografias ou documentos com capítulos.
    \begin{lstlisting}[language=tex]
    \documentclass[a4paper, 12pt]{report}
    \end{lstlisting}
    \item \verb|book|: Projetado para livros, teses e dissertações, com suporte a capítulos, partes e elementos pré-textuais.
    \begin{lstlisting}[language=tex]
    \documentclass[a4paper, 12pt]{book}
    \end{lstlisting}
    \item \verb|beamer|: Para criar apresentações de slides.
    \begin{lstlisting}[language=tex]
    \documentclass{beamer}
    \end{lstlisting}
\end{enumerate}

\section{Estrutura básica de um documento}
Todo documento em LaTeX possui três partes principais:
\begin{enumerate}
    \item \textbf{Pré-âmbulo}: Define a classe, pacotes e configurações globais.
    \item \textbf{Corpo}: Contém o conteúdo visível do documento.
    \item \textbf{Pós-âmbulo} (opcional): Inclui apêndices, glossários ou índices.
\end{enumerate}

\begin{lstlisting}[language=tex, caption=Exemplo de estrutura mínima]
    \documentclass{article} % Pré-âmbulo
    \begin{document}         % Início do corpo
        Olá, mundo!          % Conteúdo
    \end{document}           % Fim do corpo
\end{lstlisting}

\section{Elementos específicos por classe}

\subsection{Artigo (\texttt{article})}
\begin{itemize}
    \item Capa simples, resumo (\textit{abstract}), seções (sem capítulos)
    \begin{lstlisting}[language=tex, caption=Exemplo de artigo]
    \documentclass{article}
    \title{Meu Artigo}
    \author{Fulano}
    \date{\today}
    \begin{document}
        \maketitle
        \begin{abstract}
            Resumo do artigo...
        \end{abstract}
        \section{Introdução}
            Texto da introdução...
    \end{document}
    \end{lstlisting}
\end{itemize}

\subsection{Livro (\texttt{book})}

\begin{itemize}
    \item Divisão clara entre elementos pré-textuais (\verb|\frontmatter|) e o corpo (\verb|\mainmatter|).
    \begin{lstlisting}[language=tex, caption=Exemplo de livro]
    \documentclass{book}
    \title{Meu Livro}
    \author{Fulano}
    \begin{document}
        \frontmatter       % Pré-texto (numeração romana)
            \maketitle
            \tableofcontents
        \mainmatter        % Corpo (numeração arábica)
            \chapter{Introdução}
                Texto...
    \end{document}
    \end{lstlisting}
\end{itemize}

\section{Geometria da página}

Use o pacote \verb|geometry| para ajustar margens:

\begin{lstlisting}[language=tex, caption=Ajuste das margens com o pacote \texttt{geometry}]
    \usepackage[top=3cm, left=3cm, right=2cm, bottom=2cm]{geometry}
\end{lstlisting}

\section{Dicas práticas}

\begin{itemize}
    \item Pacotes essenciais:
    \begin{lstlisting}[language=tex, caption=Pacotes mais fundamentais]
    \usepackage[brazilian]{babel}     % Traduz termos para português
    \usepackage[utf8]{inputenc}      % Suporte a acentuação
    \usepackage{amsmath}             % Fórmulas matemáticas
    \end{lstlisting}
    \item Erros comuns:
    \begin{itemize}
        \item Não fechar ambientes (\verb|\begin{document} ...\end{document}|).
        \item Usar \verb|\chapter| em classes que não suportam (como \verb|article|).
    \end{itemize}
\end{itemize}
